\begin{acquis}
\begin{itemize}
    \item Définir les termes suivants : action mécanique, force, droite d'action, sens, intensité, point d'application, vecteur, système, acteur, receveur, dynamomètre, newton, résultante des forces
%
    \item Modéliser une action mécanique par une force
%
    \item Décrire quels sont les effets d'une action mécanique
%
    \item Représenter une force par un vecteur en respectant une échelle
%
    \item Identifier, pour une situation donnée, le système, l'acteur, le receveur
%
    \item Citer les caractéristiques d'une force (droite d'action, sens, intensité, point d'application)
%
    \item Établir un diagramme objets--interactions pour une situation donnée
%
    \item Déduire d'un diagramme objets--interactions les forces s'appliquant sur le système étudié
%

    \item Établir graphiquement la somme de plusieurs vecteurs
%
    \item Construire graphiquement la résultante de plusieurs forces, et déterminer son intensité en utilisant une échelle
\end{itemize}
\end{acquis}


\QCMautoevaluation{Pour chaque question, plusieurs réponses sont
  proposées. Déterminer celles qui sont correctes.}

\begin{QCM}
  
  \begin{GroupeQCM}
    
    \begin{exercice}
      La somme de deux forces d'intensité 2\,N est une force d'intensité :
      \begin{ChoixQCM}{4}
      \item $2$\,N
      \item $4$\,N
      \item $0$\,N
      \item on ne peut pas le savoir a priori.
      \end{ChoixQCM}
        \begin{corrige}
        \reponseQCM{d}
        \end{corrige}
    \end{exercice}
    
    
    \begin{exercice}
      Parmi les termes proposés quels sont ceux qui sont équivalents :
      \begin{ChoixQCM}{4}
      \item droite d'action et direction
      \item direction et sens
      \item sens et droite d'action
      \item intensité et valeur
      \end{ChoixQCM}
\begin{corrige}
     \reponseQCM{ad}
   \end{corrige}
    \end{exercice}
    
\end{GroupeQCM}
\end{QCM}



\textbf{TODO : ajouter des exercices "identifier l'action et la réaction", avec par exemple une personne qui pousse un mur, un objet qui tombe, les gaz d'une fusée. Dire qu'elle est l'interaction, quelle est l'action, quelle est la réaction}
