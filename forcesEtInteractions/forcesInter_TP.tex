\begin{TP}[Balance à élastique]



\begin{cadre}

Le but de ce TP est de trouver la méthode la plus précise permettant de déterminer la masse de deux objets mystères en étirant des élastiques. À vous de concevoir votre manipulation ! Pour cela, il faut suivre les étapes de la fiche méthode \emph{\og Concevoir une expérience \fg} disponible en annexe à la fin de ce manuel.

\end{cadre}

\vspace{1em}



\begin{partie}[Conception et réalisation de l'expérience]

Vous devez concevoir une expérience permettant d'apporter une réponse au problème posé. L'analyse des données recueillies consistera en un graphique représentant l'évolution de la variable dépendante en fonction de la variable indépendante.

Matériel à disposition :
\begin{itemize}
\item élastiques ;
\item support ;
\item boîte de masses marquées ;
\item feuille de papier millimétré.
\end{itemize}

\end{partie}

\vspace{1em}

\begin{partie}[Détermination de la masse des deux objets mystères]

Lorsque vous êtes prêt, appelez le professeur pour déterminer la masse des deux objets mystères !

Matériel à disposition :
\begin{itemize}
\item élastiques ;
\item support ;
\item résultats de l'étape 1.
\end{itemize}

\end{partie}

\vspace{3em}

\textbf{Évaluation}

\vspace{1em}

Vous devez rendre un compte rendu expliquant la démarche suivie, donnant vos résultats et leur analyse. Vous devez expliquez comment vous avez pu déterminer la masse des objets mystères à l'aide de vos résultats.

\vspace{1em}

Grille d'évaluation du compte rendu :
\begin{itemize}
\item les 5 étapes de la démarche scientifique apparaissent et sont correctement développées ;
\item variables choisies (description indépendante, dépendante, et unités) ;
\item protocole détaillé de l'expérience réalisée ;
\item schéma légendé du montage utilisé ;
\item données récoltées (volume, gamme, répartition, tableau de valeur, unités) ;
\item traitement des données (graphique : titre, axes, grandeurs, unités, échelle, qualité, propreté, justesse) ;
\item détermination de la masse des objets mystères et explication de la méthode utilisée ;
\item globalisation (soin, propreté, investissement, qualité générale).
\end{itemize}





\end{TP}
