% commandes utilisées pour la physiques


% divers
\providecommand{\abs}[1]{\lvert#1\rvert}	% pour avoir les valeurs absolues
\providecommand{\norme}[1]{\lVert#1\rVert}	% pour avoir les normes en mode math
\providecommand{\vect}[1]{\overrightarrow{#1}}	% les vecteurs en mode math
\newcommand{\ee}[1]{$\times$10$^{{#1}}$}	% x10^argument
\newcommand{\eem}[1]{\times 10^{{#1}}}		% x10^argument en mode math

% physique : unités
\newcommand{\vit}{\,m$\cdot$s$^{-1}$\xspace} 		% m.s-1
\newcommand{\vitk}{\,km$\cdot$h$^{-1}$\xspace}		% km.h-1
\newcommand{\acc}{\,m$\cdot$s$^{-2}$\xspace} 		% m.s-2
\newcommand{\vitr}{\,rad$\cdot$s$^{-1}$\xspace} 	% rad.s-1

\newcommand{\unitun}[2]{\,{#1}$^{{#2}}$\xspace} 	% arg1^arg2
\newcommand{\unitdeux}[3]{\,{#1}$\cdot${#2}$^{{#3}}$\xspace} % arg1.arg2^arg3
\newcommand{\unittrois}[4]{\,{#1}$\cdot${#2}$\cdot${#3}$^{{#4}}$\xspace} % arg1.arg2.arg3^arg4

% physique : dérivées
\newcommand{\dt}[1]{\frac{\text{d}{#1}}{\text{d}t}}	% en mode math d(arg1)/dt
\newcommand{\ddt}[1]{\frac{\text{d}^2{#1}}{\text{d}t^2}}	% en mode math d2(arg1)/dt2

% physique : vecteur à 2 et à 3 composantes et les accolades pour 2 et 3 lignes
\newcommand{\trinome}[3]{\left(\begin{matrix}{#1}\\{#2}\\{#3}\\ \end{matrix}\right)} %id. binom
\newcommand{\binome}[2]{\left(\begin{matrix}{#1}\\{#2}\\ \end{matrix}\right)} %id. binom
\newcommand{\binacc}[2]{\left\lbrace\begin{array}{l}{#1}\\{#2}\\ \end{array}\right.} % accolade
\newcommand{\trinacc}[3]{\left\lbrace\begin{array}{l}{#1}\\{#2}\\{#3}\\ \end{array}\right.} %ac


% thermodynamique
\newcommand{\jmol}{\,J$\cdot$mol$^{-1}$\xspace} 	% J.mol-1
\newcommand{\kjmol}{\,kJ$\cdot$mol$^{-1}$\xspace} 	% kJ.mol-1
\newcommand{\dhf}[1]{\Delta \text{H}^\text{o}_\text{f}(\text{#1})}  % delta H° formation (X)
\newcommand{\dhfr}{\Delta \text{H}^\text{o}_\text{f}}  % delta H° formation
\newcommand{\dhr}{\Delta \text{H}^\text{o}_\text{R}}  % delta H° réaction
\newcommand{\dho}{\Delta \text{H}^\text{o}}  % delta H°
\newcommand{\dso}{\Delta \text{S}^\text{o}}  % delta S°
\newcommand{\dgo}{\Delta \text{G}^\text{o}}  % delta G°
\newcommand{\dsr}{\Delta \text{S}^\text{o}_\text{R}}  % delta S° réaction

% chimie
\newcommand{\mm}{\,g$\cdot$mol$^{-1}$\xspace} 		% g.mol-1
\newcommand{\conc}{\,mol$\cdot$L$^{-1}$\xspace}	% mol.L-1
\newcommand{\concmass}{\,g$\cdot$L$^{-1}$\xspace}	% g.L-1
\newcommand{\condulambda}{\,S$\cdot$m$^{2}\cdot$mol$^{-1}$\xspace} % S.m2.mol-1

