\annexe{Concevoir une expérience}

Le but est d'effectuer une \textbf{démarche scientifique} : \newline
(i) Formuler un problème, une question de recherche, (ii) Émettre une hypothèse et prévoir les conséquences observables, (iii) Tester l'hypothèse par l'expérimentation (concevoir une expérience), (iv) Confronter les résultats à la prévision, (v) Conclure (l'hypothèse est validée, l'hypothèse est confortée, l'hypothèse est rejetée).

\section{Définir le problème et sélectionner les variables}  


\begin{enumerate}
\item Définir la \textbf{question de recherche} : c'est le problème sur lequel vous allez travailler. L'hypothèse (voir plus bas) est une réponse possible à cette question. La vraie réponse sera donnée dans la conclusion.
%
\item Définir quelle sera la \underline{variable indépendante} et préciser son unité. C'est la variable, le paramètre, que vous allez faire varier dans votre expérience (cause).
%
\item Définir quelle sera la \underline{variable dépendante} et préciser son unité. C'est la variable, le paramètre, que vous allez mesurer dans votre expérience (effet).
%
\item Lister ce que vous savez déjà sur le sujet : recherches effectuées, connaissances personnelles, observations déjà effectuées, premiers essais. Ceci doit vous permettre de formuler une hypothèse de recherche valable.
%
\item Écrire une \textbf{hypothèse} pour votre expérience, basée sur ce que vous savez déjà. L'hypothèse est une \textbf{prédiction du résultat de l'expérience}. Les variables indépendante et dépendante doivent clairement apparaître dans l'hypothèse. L'hypothèse doit pouvoir être testée et vérifiée par l'expérience.
%
\item Précisez et détaillez comment sont mesurées les variables indépendante et dépendante.
%
\item Identifier les \underline{variables contrôlées} et leurs effets possibles sur les résultats. Ce sont les variables, les paramètres de l'expérience, qui doivent rester constants afin de ne pas fausser les résultats. Décrire quelles seront vos actions permettant de contrôler ces variables. 
%
\item Proposer une \textbf{expérience}. Détailler pas à pas les étapes à réaliser pour mener à bien votre expérience (première version du protocole d'expérience). Les étapes doivent être suffisamment claires pour permettre à une autre personne de réaliser la même expérience. Lister précisément le matériel et les produits nécessaires. Remarque : c'est quand vous rédigerez votre compte rendu, une fois l'expérience réalisée, que vous rédigerez la version finale du protocole d'expérience.
%
\end{enumerate}

\section{Avant de collecter les données}

\begin{enumerate}
%
\item Assurez-vous que suffisamment de données seront collectées (quel volume de données est nécessaire pour valider votre hypthèse ?).
\item Combien de fois seront répétées les mesures afin de s'assurer de la reproductibilité des résultats et améliorer leur précision ?
\item Préparer un grand tableau pour y consigner les résultats bruts. Écrire un titre descriptif au sommet de chaque tableau. Numéroter chacun des tableaux, afin de pouvoir s'y référer (\emph{<<\,voir tableau \no 1\,>>}). Inclure les unités et les incertitudes de mesure pour chaque colonne dans les tableaux de données.
\item Faire un schéma ou prendre une photographie du montage utilisé. Légender tout l'équipement et le matériel utilisés.
\end{enumerate}





\section{Traiter les données}

\textsl{Le traitement des données inclut tous les calculs ou manipulations réalisés sur les données brutes : calculs variés, réalisation de graphiques, etc. En général, la variable indépendante est représentée en abscisse et la variable dépendante en ordonnée.}

\begin{enumerate}
\item Inclure un exemple de chaque calcul effectué sur les données bruts (vérifier les calculs pour éviter les erreurs !).
\item Présenter les résultats traités sous forme de feuille de calculs, tableaux, graphiques, illustrations, diagrammes, etc.
\item Les graphiques doivent être précis et posséder des étiquettes d'axes avec unités.
\item Sous chaque graphique, illustration, diagramme, etc., inclure une numérotation (permettant de s'y référer) et un titre qui décrit ce qui est présenté.

\item Les valeurs numériques doivent être présentés avec une unité, un nombre de chiffres significatifs adaptés et une incertitude lorsque cela est pertinent.
\end{enumerate}




\section{Apporter une réponse}

\textsl{Vous devez apporter une réponse à votre question de recherche.}


\begin{enumerate}
\item Interprétez vos résultats : quelle est la réponse à votre problème ? Relisez votre hypothèse et comparez votre réponse à ce que vous aviez prévu. Les données collectées sont-elles en accord avec vos hypothèses ? Pourquoi ou pourquoi pas ?

\item Vos résultats sont-ils fiables ? Y a-t-il des résultats anormaux ? Quelle peut en être la cause ou la signification ?

\item Comparez vos résultats à d'autres sources de données ou d'informations (citez les sources). Calculez l'écart en pourcent entre vos valeurs et celles de la littérature. Tirez une conclusion.

\end{enumerate}



\section{Évaluation et amélioration}

\begin{enumerate}
\item Avez-vous rencontrer des problèmes lors de vos expériences ? Comment ont-ils affecté vos résultats ? Qu'avez-vous fait pour limiter leurs impacts sur les résultats ?

\item Quels sont les points faibles de votre protocole et comment ont-ils pu affecter vos résultats ? Pour chaque point faible ou limitation mentionné, proposez une solution ou des pistes d'amélioration.

\item Est-ce que quelque chose pouvant affecter la fiabilité de vos résultats est arrivé durant vos expériences ?

\item Comment pourriez-vous améliorer, développer ou approfondir cette recherche en vue d'une étude future ?
\end{enumerate}

\vspace{1em}
Pour rédiger le compte rendu, vous pouvez utiliser la fiche méthode \emph{Rédiger un compte rendu d'expérience}. 

{\huge todo : faire un lien avec la fiche dem sciQ et la fiche rédiger comt-rendu}
