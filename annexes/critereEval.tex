\annexe{Les critères d'évaluation}

\vspace{3em}
%%%%%%%%%%%%%%%%%%%%%%%%%%%%%%%%%%%%%%%%%%%%%%%%%%%%%%%%%%%%%%%%%
%                   TABLEAU CRITÈRES
\begin{center}
{\footnotesize % fin du footnotesize 11 lignes plus bas
\definecolor{gris}{rgb}{.7,.7,.7}
\begin{tabularx}{.7\linewidth}{|l|X|c|c|}
\cline{3-4}
\multicolumn{2}{l|}{} & {\scriptsize Points obtenus} & {\scriptsize Total des points}    \\ \hline
Critère A   & \textsl{Connaissance et compréhension} & & \\ \hline
Critère B   & Démarche scientifique : \textsl{Recherche et élaboration}      & & \\ \hline
Critère C   & \textsl{Communication}                 & & \\ \hline
Critère D   & Démarche scientifique : \textsl{Traitement et évaluation}      & & \\ \hline
\multicolumn{2}{l|}{}   & \cellcolor{gris} & \cellcolor{gris} \\ \cline{3-4}
\end{tabularx}
} % fin du footnotesize
\end{center}
%
%%%%%%%%%%%%%%%%%%%%%%%%%%%%%%%%%%%%%%%%%%%%%%%%%%%%%%%%%%%%%%%%%



\section{Critère A (Connaissance et compréhension)}

L'élève est capable :
\begin{itemize}
\item de répondre à des questions de cours, de citer des définitions, de refaire un schéma ;
\item d’expliquer des connaissances scientifiques ;
\item d’appliquer des connaissances et une compréhension scientifiques pour résoudre des problèmes tirés de situations aussi bien familières que nouvelles ;
\item d’analyser et d’évaluer des informations afin de formuler des jugements scientifiquement étayés.
\end{itemize}


\section{Critère B (Démarche scientifique : recherche et élaboration)}

L'élève est capable :
\begin{itemize}
\item d’expliquer un problème ou une question qui sera vérifié(e) par une recherche scientifique ;
\item de formuler une hypothèse vérifiable et de l’expliquer en faisant appel à un raisonnement scientifique ;
\item d’expliquer la façon de manipuler les variables et d’expliquer la manière dont les données seront recueillies ;
\item d’élaborer des recherches scientifiques.
\end{itemize}



\section{Critère C (Communication)}

L'élève est capable :
\begin{itemize}
\item de rendre compte de façon écrite, de manière synthétique et structurée, en utilisant un vocabulaire adapté, une langue correcte et précise ;
\item de rendre compte de façon orale, de résumer sa démarche, de transmettre l’information de manière synthétique et claire, de s’exprimer à l’oral avec aisance ;
\item d'utiliser un langage scientifique de manière correcte et efficace ;
\item de présenter des résultats avec l'outil informatique ;
\item de documenter les travaux d’autrui et les sources d’information utilisées.
\end{itemize}


\section{Critère D (Démarche scientifique : traitement et évaluation)}

L'élève est capable :
\begin{itemize}
    \item de présenter des données recueillies et transformées ;
    \item d’interpréter des données et d’expliquer des résultats en faisant appel à un raisonnement scientifique ;
    \item d’évaluer la validité d’une hypothèse en fonction du résultat de la recherche scientifique ;
    \item d’évaluer la validité de la méthode employée ;
    \item d’expliquer des moyens d’améliorer ou d’approfondir la méthode.
\end{itemize}






